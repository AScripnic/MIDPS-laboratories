\section*{Concluzie}
\phantomsection

Pe parcursul elaborării lucrării date am studiat un IDE nou Unity, care permite lansare aplicațiilor pe o listă mare de platforme, ca ex: Android, Windows, iOS, OSx, Playstation etc., ce oferă un API foarte user friendly pentru developerii care nu au mai lucrat întru-un asemenea domeniu. 

Am profitat de avantajele sistemului de control a versiunilor, git, în momentul când am șters content de care aveam nevoie.

Am avut posibilitatea de a învăța diferențele dintr-un limbaj tipizat, C\#, și unul netipizat, UnityScript, și avantajele și dezavantajele fiecărui în parte.

Am învățat care sunt bazele unui joc și care sunt pașii obligatorii care trebuie executați pe parcursul creării unui joc, din greșelile admise de mine:
\begin{itemize}
	\item Căutarea texturilor se face înainte de a începe lucrul asupra jocului
	\item Căutarea texturilor se face pentru tot jocul nu doar pentru personaj (în cazul în care nu ai designer skills)
	\item Dimensiunile obiectelor se fac pe bază de forma care o va avea după aplicarea texturilor
	\item Se studiază limbajele de programarea disponibile pentru un IDE apoi se începe lucrul, asa cum unul poate avea avantaje mai mare în cadrul unui IDE decât altele
\end{itemize}


\clearpage